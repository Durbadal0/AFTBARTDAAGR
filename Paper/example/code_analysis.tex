\documentclass{article}
\usepackage{amsmath, amssymb}

\begin{document}

\section*{Model Specification}
We assume that for each observation $i$, we observe $y_i$ (possibly censored) and an indicator $\Delta_i$ where:
\[
\Delta_i = \begin{cases}
1, & \text{if } y_i \text{ is observed (uncensored)};\\[1mm]
0, & \text{if } y_i \text{ is censored.}
\end{cases}
\]
We work on the log-scale by defining a latent variable 
\[
\tilde{y}_i = \log(y_i) \quad \text{(or an adjusted value for censored observations).}
\]
The regression model is
\[
\tilde{y}_i = X_i \beta + r_{c(i)} + \varepsilon_i, \quad \varepsilon_i \sim N(0,\sigma^2),
\]
where:
\begin{itemize}
  \item $X_i$ is the row vector of covariates for observation $i$.
  \item $\beta$ is the vector of fixed effects.
  \item $r_{c(i)}$ is the random effect associated with the cluster $c(i)$ that observation $i$ belongs to.
  \item $\sigma^2$ is the error variance.
\end{itemize}

The random effects for each cluster $j=1,\ldots,I$ have a spatial structure through a weight matrix $W$. Let $D_j = \sum_{k=1}^{I} W_{jk}$ be the degree (row sum) for cluster $j$. A spatial prior is imposed by further introducing:
\begin{itemize}
  \item $\tau^2$: spatial variance.
  \item $\rho$: spatial correlation parameter satisfying $\rho \in [\rho_{\text{lower}},\rho_{\text{upper}}]$.
\end{itemize}

\section*{Gibbs Sampler Steps}

\subsection*{Step 1: Imputation of Latent Variables}
For each observation $i$, if $\Delta_i = 0$, $\tilde{y}_i$ is missing (censored) and is imputed from a truncated normal distribution:
\[
\tilde{y}_i \sim TN\left(a = \log(y_i), \, b = \infty; \, \mu_i, \, \sigma^2\right),
\]
with
\[
\mu_i = X_i \beta + r_{c(i)}.
\]
For uncensored observations ($\Delta_i=1$), we set 
\[
\tilde{y}_i = \log(y_i).
\]

\subsection*{Step 2: Update of $\beta$ (Fixed Effects)}
Define an adjusted outcome
\[
y_i^* = \tilde{y}_i - r_{c(i)}.
\]
Assuming a conjugate normal prior for $\beta$, that is,
\[
\beta \sim N(0, V_\beta) \quad \text{with } V_\beta=\text{diag}(\beta_{\text{prior_var}}),
\]
the posterior for $\beta$ is
\[
\beta \mid \tilde{y}, r, \sigma^2 \sim N\left(\mu_\beta,\, V_{\beta}^{\text{post}}\right),
\]
where
\[
V_{\beta}^{\text{post}} = \left(\frac{X^TX}{\sigma^2} + V_\beta^{-1} \right)^{-1}, \quad
\mu_\beta = V_{\beta}^{\text{post}} \left(\frac{X^T y^*}{\sigma^2}\right).
\]

\subsection*{Step 3: Update of Random Effects $r$ for Each Cluster}
For each cluster $j$, let $C_j$ be the set of observation indices belonging to cluster $j$, and let $n_j = |C_j|$. The full conditional for the random effect $r_j$ is:
\[
r_j \mid \cdot \sim N\left(m_j,\, v_j\right),
\]
with
\[
v_j = \left(\frac{n_j}{\sigma^2} + \frac{D_j}{\tau^2} \right)^{-1}, \quad
m_j = v_j\left( \frac{1}{\sigma^2} \sum_{i \in C_j} \left(\tilde{y}_i - X_i \beta\right) + \frac{\rho}{\tau^2} \sum_{k=1}^{I} W_{jk} r_k \right).
\]

\subsection*{Step 4: Update of $\sigma^2$ (Error Variance)}
The residual for each observation is given by
\[
e_i = \tilde{y}_i - X_i\beta - r_{c(i)}.
\]
An inverse-gamma prior is assumed for $\sigma^2$. The posterior distribution is
\[
\sigma^2 \sim \text{Inv-Gamma}\left(a_{\sigma} + \frac{n}{2}, \; b_{\sigma} + \frac{1}{2} \sum_{i=1}^n e_i^2 \right).
\]

\subsection*{Step 5: Update of $\tau^2$ (Spatial Variance)}
Define a quadratic form involving the random effects:
\[
Q = \sum_{j=1}^{I} D_j r_j^2 - \rho \, \mathbf{r}^T W \mathbf{r}.
\]
Then, with an inverse-gamma prior for $\tau^2$, the full conditional is
\[
\tau^2 \sim \text{Inv-Gamma}\left(a_{\tau} + \frac{I}{2}, \; b_{\tau} + \frac{Q}{2}\right).
\]

\subsection*{Step 6: Update of $\rho$ (Spatial Correlation Parameter)}
A Metropolis--Hastings step is used for $\rho$. A proposal $\rho_{\text{prop}}$ is drawn from:
\[
\rho_{\text{prop}} \sim N(\rho, \, \sigma_{\text{proposal}}^2),
\]
with $\sigma_{\text{proposal}} = 0.1$. If $\rho_{\text{prop}}$ is outside the allowable range $[\rho_{\text{lower}},\rho_{\text{upper}}]$, then the proposal is rejected. Otherwise, compute the quadratic forms:
\[
Q(\rho) = \sum_{j=1}^{I} D_j r_j^2 - \rho \, \mathbf{r}^T W \mathbf{r},
\]
and
\[
Q(\rho_{\text{prop}}) = \sum_{j=1}^{I} D_j r_j^2 - \rho_{\text{prop}} \, \mathbf{r}^T W \mathbf{r}.
\]
The log acceptance ratio is then given by:
\[
\log \alpha = -\frac{1}{2 \tau^2} \left[ Q(\rho_{\text{prop}}) - Q(\rho) \right].
\]
Accept $\rho_{\text{prop}}$ with probability $\min(1, \exp(\log \alpha))$, otherwise retain the current $\rho$.

\section*{Summary}
At each iteration, the sampler:
\begin{enumerate}
    \item Imputes latent values $\tilde{y}_i$ for censored observations.
    \item Updates the fixed effects $\beta$ via a conjugate normal posterior.
    \item Updates the cluster-level random effects $r_j$ based on the observed data and the spatial structure.
    \item Updates $\sigma^2$ and $\tau^2$ using inverse-gamma full conditionals.
    \item Updates $\rho$ using a Metropolis--Hastings step.
\end{enumerate}
Posterior samples for all these parameters are stored for later inference.

\end{document}